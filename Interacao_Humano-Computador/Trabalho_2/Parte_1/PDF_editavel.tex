\documentclass[a4paper,12pt]{article}
\usepackage[brazil]{babel}
\usepackage[utf8]{inputenc}
\usepackage{lmodern}
\usepackage{geometry}
\usepackage{graphicx}
\geometry{margin=2.5cm}
\title{Perfis, Personas e Cenários \\ \large Plataforma de Doação de Roupas e Alimentos}
\author{}
\date{}

\begin{document}

\begin{figure}
    \centering
    \includegraphics[width=0.25\linewidth]{Brasao4_vertical_cor_300dpi.png}
    \caption{Brasão UFC}
    \label{fig:enter-label}
\end{figure}

\begin{center}
    \textbf{Universidade Federal do Ceará – UFC \\
    Campus Quixadá}
    
    \vspace{0.3cm}
    
    Esta pesquisa foi desenvolvida para a disciplina de Interação Humano-Computador (IHC), como parte da proposta de criação de uma plataforma de cadastro de doadores de roupas e de alimentos não perecíveis.

    \vspace{0.3cm}

    Trabalho elaborado pelos alunos do curso de Ciência da Computação: \\
    Lafuente Paulino da Silva, Patrick de Farias Ramos e Nicolas Ferreira Leite.
\end{center}

\vspace{1cm}

\maketitle

\section{Perfil de Usuário - Público Doador}

\begin{itemize}
    \item \textbf{Percentual:} 55\% dos usuários analisados
    \item \textbf{Faixa etária:} 18 a 24 anos
    \item \textbf{Gênero predominante:} Masculino (60\%), mas presença relevante feminina
    \item \textbf{Ocupação:} Estudantes universitários
    \item \textbf{Tempo de experiência com tecnologia:} Alta: utiliza internet e aplicativos diariamente
    \item \textbf{Atitude perante tecnologia:} Aprecia tecnologia, considera-se adaptado
    \item \textbf{Estilo de aprendizado:} Aprende explorando aplicativos e tutoriais online
    \item \textbf{Frequência de doação:} Esporádica: quando lembra ou tem excesso
    \item \textbf{Conhecimento sobre plataformas de doação:} Baixo: maioria nunca utilizou apps ou sites de doação
    \item \textbf{Principais dificuldades:} Falta de informação sobre onde doar e desconfiança em instituições
    \item \textbf{Preferência:} Doação anônima, com praticidade e agilidade
\end{itemize}

\section{Persona - João Pedro}

\begin{itemize}
    \item \textbf{Identidade:} João Pedro, 21 anos
    \item \textbf{Status:} Persona primária (usuário típico)
    \item \textbf{Ocupação:} Estudante de Engenharia de Software
    \item \textbf{Localização:} Quixadá, Ceará
    \item \textbf{Objetivos:} Ajudar pessoas de forma prática e anônima; liberar espaço em casa sem desperdiçar itens
    \item \textbf{Habilidades:} Uso fluente de smartphones, redes sociais, apps de transporte e compras online
    \item \textbf{Tarefas básicas:} Localizar instituições confiáveis; agendar ou encontrar pontos de coleta; doar rapidamente
    \item \textbf{Tarefas críticas:} Garantir que sua doação chegue a quem precisa com segurança
    \item \textbf{Relacionamentos:} Colegas de república, amigos e influências de redes sociais
    \item \textbf{Requisitos:} Plataforma intuitiva, com informações claras sobre instituições; opção de anonimato; lembretes automáticos
    \item \textbf{Expectativas:} Processo rápido, seguro, transparente e que não exija muito tempo ou deslocamento
\end{itemize}

\section{Cenário - João Pedro utilizando a Plataforma de Doações}

\begin{itemize}
    \item \textbf{Ambiente ou contexto:} João organiza seu guarda-roupa, separa roupas e alimentos, e busca uma maneira de doá-los.
    \item \textbf{Atores:} João Pedro (doador); Plataforma de Doações; Instituições de caridade
    \item \textbf{Objetivos:} Encontrar instituição confiável e próxima; doar de forma rápida e anônima
    \item \textbf{Planejamento:} Busca na internet por opções, encontra a plataforma de doação
    \item \textbf{Ações:} 
    \begin{enumerate}
        \item Acessa a plataforma pelo celular
        \item Realiza cadastro com opção de anonimato
        \item Consulta mapa de pontos de coleta
        \item Agenda entrega
        \item Recebe lembrete e faz a doação
    \end{enumerate}
    \item \textbf{Eventos:} Recebe feedback e comprovante digital da doação
    \item \textbf{Avaliação:} Sente-se satisfeito e motivado a continuar utilizando a plataforma
\end{itemize}

\newpage

\section{Perfil de Usuário - Público Beneficiário (em situação de vulnerabilidade)}

\begin{itemize}
    \item \textbf{Percentual:} 100\% dos entrevistados
    \item \textbf{Faixa etária:} De 30 a 50 anos
    \item \textbf{Gênero predominante:} Indefinido (anônimo)
    \item \textbf{Ocupação:} Desempregados ou trabalhadores informais
    \item \textbf{Tempo de experiência com tecnologia:} Baixa: ausência de telefone celular ou acesso restrito à internet
    \item \textbf{Atitude perante tecnologia:} Neutra: não por resistência, mas por falta de recursos
    \item \textbf{Estilo de aprendizado:} Prático, aprendem na interação direta com pessoas e serviços comunitários
    \item \textbf{Acesso a alimentos e roupas:} Por meio de doações ocasionais, ajuda comunitária, programas sociais
    \item \textbf{Conhecimento sobre plataformas de doação:} Nulo: não conhecem ou não acessam
    \item \textbf{Principais dificuldades:} Falta de acesso regular a alimentos e vestuário; ausência de renda fixa
    \item \textbf{Preferência:} Locais físicos acessíveis, como pontos de coleta e organizações comunitárias
\end{itemize}

\section{Persona - Pessoa Anônima em Situação de Vulnerabilidade}

\begin{itemize}
    \item \textbf{Identidade:} Anônimo, 40 anos
    \item \textbf{Status:} Persona primária (usuário beneficiado)
    \item \textbf{Ocupação:} Desempregado, realizando pequenos serviços informais
    \item \textbf{Localização:} Quixadá, Ceará
    \item \textbf{Objetivos:} Garantir acesso regular a alimentos e roupas; melhorar condição de vida; restabelecer vínculos familiares
    \item \textbf{Habilidades:} Baixa escolaridade; não possui familiaridade com tecnologia por falta de acesso
    \item \textbf{Tarefas básicas:} Buscar refeições em espaços comunitários; solicitar doações pessoais; realizar pequenos serviços
    \item \textbf{Tarefas críticas:} Garantir segurança alimentar e vestuário adequados
    \item \textbf{Relacionamentos:} Comunidade local, voluntários, instituições beneficentes
    \item \textbf{Requisitos:} Facilidade de acesso a pontos físicos de coleta; campanhas de sensibilização
    \item \textbf{Expectativas:} Existência de locais públicos organizados e permanentes para doações
\end{itemize}

\section{Cenário - Pessoa Anônima utilizando a ajuda promovida pela plataforma}

\begin{itemize}
    \item \textbf{Ambiente ou contexto:} A pessoa circula pelo centro de Quixadá buscando comida e roupas; ouve sobre um novo ponto de coleta promovido pela plataforma
    \item \textbf{Atores:} Pessoa em vulnerabilidade; Comunidade local; Plataforma de Doações; Instituições locais
    \item \textbf{Objetivos:} Conseguir roupas limpas e alimentação
    \item \textbf{Planejamento:} Dirige-se a um centro comunitário onde costuma receber refeições
    \item \textbf{Ações:}
    \begin{enumerate}
        \item Vai até o centro comunitário
        \item Recebe uma refeição
        \item Descobre novo ponto de distribuição de roupas
        \item Recebe roupas adequadas
    \end{enumerate}
    \item \textbf{Eventos:} A comunidade começa a doar mais regularmente; os pontos de coleta são mantidos
    \item \textbf{Avaliação:} Sente alívio e gratidão; percebe a importância da rede de solidariedade
\end{itemize}

\section{Cenário de Falha - Dificuldade no acesso às doações}

\begin{itemize}
    \item \textbf{Ambiente ou contexto:} A pessoa em situação de vulnerabilidade busca ajuda, mas os pontos de coleta estão desatualizados ou desabastecidos.
    \item \textbf{Atores:} Pessoa beneficiada; Plataforma de Doações; Instituições locais
    \item \textbf{Objetivos:} Encontrar alimento e vestuário
    \item \textbf{Planejamento:} Vai ao centro comunitário baseado em informações antigas
    \item \textbf{Ações:}
    \begin{enumerate}
        \item Dirige-se ao local
        \item Descobre que o ponto está desativado ou vazio
        \item Sente frustração e insegurança
    \end{enumerate}
    \item \textbf{Eventos:} Falta de atualização no sistema; falha de comunicação entre plataforma e instituições
    \item \textbf{Avaliação:} Percepção negativa; necessidade urgente de melhoria na gestão de pontos físicos de doação
\end{itemize}

\end{document}
